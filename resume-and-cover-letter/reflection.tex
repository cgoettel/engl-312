\documentclass{article}
\usepackage[margin=1in]{geometry}

\usepackage{setspace}
\linespread{2}

\begin{document}
\begin{center}
    {\Large Reflection}
\end{center}

% a. If you chose to ignore any of the guidelines for resumes and cover letters, why did you make the choices you did? How do you think your audience will respond?
I chose to ignore using bullet points on my resume to list off work experience. Previous to this class, I rarely saw bullet points. I prefer prose and think, from a typographic standpoint, that it looks better and flows better. In workshop on Wednesday, the other group criticized it (the only criticism on the resume), but I feel like this is because they haven't been taught about resumes from more than one teacher. Not that the teaching is wrong, but that it's not concrete.

The other criticism I got was that the letter of intent was too long. This criticism was because the other group assumed it was merely a cover letter. The graduate school requirements specify this letter as being 1--2 pages and covering each of the sections that I did.

% b. What feedback from your Career Center mentor or peers was most useful and how did you implement it?
The letter of intent for graduate school went through many serious revisions. I went over it with my dad (Deputy CFO at JPL, previous PhD student, two master's degrees) and he heavily critiqued both the style and content. Removing weaknesses and portraying them in a positive light is the best improvement I was able to incorporate.

My resume has been reviewed by my previous boss, both of my academic advisors, and countless students in the program. It has been being shaped for years. The resume I've turned in is something that I'm proud of because of all the work I've had to put into it.

% c. If you magically had an extra two weeks to complete this portfolio, what would you change or improve and why/how?
If I had an extra two weeks to complete this I wouldn't do it any differently. I've previously taken a class built around building resumes and cover letters in the IT industry, I've used that knowledge to build my resume, to portray myself in a positive and impressive light, and to get a job. Just this semester I've applied to twelve jobs, tailoring resumes and cover letters for each.

\end{document}
