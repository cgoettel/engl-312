\documentclass[12pt]{article}
\usepackage[margin=1in]{geometry}

\usepackage{fontspec}
\setmainfont[BoldFont={HoeflerText-Bold},ItalicFont={Hoefler Text Italic},BoldItalicFont={Hoefler Text Black Italic},SmallCapsFont={HoeflerText-RomanSC},Ligatures={NoCommon, TeX}]{Hoefler Text}
\setmonofont[Ligatures={TeX}]{Source Code Pro Medium}

\usepackage[normalem]{ulem}

\title{Marrying Your Sister}
\author{Colby Goettel}

\begin{document}
\maketitle

% 1) Audience Analysis
% Write one or two brief paragraphs analyzing your target audience and their values, beliefs, and assumptions. You are writing your paper for a general American audience: imagine that your argument will be distributed in a national newspaper or magazine. Within this general audience, you will need to identify who constitutes your specific audience. Who might find your claim initially unacceptable? What kinds of groups or individuals would care about your issue (e.g., educators, health care providers, government officials, parents, etc.)? What values, beliefs, or assumptions do they hold about your issue? Do you personally know any individuals who would disagree with your stance? What would their probable response to your argument be?

% target audience: 
% audience values: 



% 2) Consequential Issue Question
% Write out the WATCO question (What are the consequences of A on B?) that your paper will answer.
What are the consequences of 

% 3) Pro/Con Analysis
% Summarize the arguments for and against your issue. You can write this pro/con analysis in the form of a bulleted list or as a paragraph.
Pros: 

Cons: 

% 4) Preliminary Enthymeme
% Draft a preliminary enthymeme for your paper. This should include your claim, reason, and your assumption. Please follow the format below for writing your enthymeme (be sure to mark the terms and underline the verbs as shown):
 (A) \uline{} (v1)  (B) because  (A) \uline{} (v2) .

% Banning the advertising of junk food to children (A) decreases (v1) childhood obesity rates (B)
% b/c
% Banning the advertising of junk food to children (A) decreases (v2) calorie consumption from junk food among children (C).

\end{document}
