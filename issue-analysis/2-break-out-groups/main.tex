\documentclass[12pt]{article}
\usepackage[margin=1in]{geometry}

\usepackage{fontspec}
\setmainfont[BoldFont={HoeflerText-Bold},ItalicFont={Hoefler Text Italic},BoldItalicFont={Hoefler Text Black Italic},SmallCapsFont={HoeflerText-RomanSC},Ligatures={NoCommon, TeX}]{Hoefler Text}
\setmonofont[Ligatures={TeX}]{Source Code Pro Medium}

\usepackage[normalem]{ulem}

\title{Using Breakout Groups in Church Teaching}
\author{Colby Goettel}

\begin{document}
\maketitle

% 1) Audience Analysis
% Write one or two brief paragraphs analyzing your target audience and their values, beliefs, and assumptions. You are writing your paper for a general American audience: imagine that your argument will be distributed in a national newspaper or magazine. Within this general audience, you will need to identify who constitutes your specific audience. Who might find your claim initially unacceptable? What kinds of groups or individuals would care about your issue (e.g., educators, health care providers, government officials, parents, etc.)? What values, beliefs, or assumptions do they hold about your issue? Do you personally know any individuals who would disagree with your stance? What would their probable response to your argument be?

% target audience: teachers
% audience values: effective teaching

Teachers have a deep-seated desire for students to learn. Because of this, they are constantly looking for the best ways to teach~--- ways that stimulate the minds of all of the students, not just a select few. Teachers teach for the mass, but also focus on the one. It is a difficult balance to focus on each individual while still being able to help the entire class learn and grow together. In a gospel setting, teachers want their students to feel the spirit because that is the one sure way that they will all learn. Then, how can teachers best bring in the spirit while teaching? When teachers have control of the class discussion, they can guide the discussion in the way the spirit is telling them to. It is their stewardship and they are able to receive that necessary revelation. However, when they lose control of the situation, any topic~--- tangential or otherwise~--- can come up and the spirit can quickly leave the room. The teaching has become ineffective.

When breakout groups each read different sections of scripture, the other students don't know what happened in that section. Even with a brief summary, it's not as effective as focusing the entire class on one section of scripture, reading it together, discussing it. When breakout groups are set up, time is wasted, and students are rarely set up in properly sized and oriented groups. Groups are generally too large to encourage personal and thoughtful discussion. Groups are usually set up in columns of students, not clusters. This means that the students who are supposed to be discussing an issue together are generally the worst suited for a group discussion without having to move their seats. And moving seats is another time-sink. Students care about learning, but also the social aspects of church (girls, etc.).

% 2) Consequential Issue Question
% Write out the WATCO question (What are the consequences of A on B?) that your paper will answer.
What are the consequences of using breakout groups on student learning?

% 3) Pro/Con Analysis
% Summarize the arguments for and against your issue.
Pros: (theoretically) everyone is involved in breakout groups, better learning without, more control of discussion, better spirit in the room, more effective use of resources.

Cons: distraction, students not participating, students not understanding what other students read, class size can be a major issue.

% 4) Preliminary Enthymeme
% Draft a preliminary enthymeme for your paper. This should include your claim, reason, and your assumption. Please follow the format below for writing your enthymeme (be sure to mark the terms and underline the verbs as shown):
Using breakout groups (A) \uline{stymies} (v1) student learning (B) because using breakout groups (A) \uline{makes} (v2) ineffective use of class time.

% Banning the advertising of junk food to children (A) decreases (v1) childhood obesity rates (B)
% b/c
% Banning the advertising of junk food to children (A) decreases (v2) calorie consumption from junk food among children (C).

\end{document}
