\documentclass[12pt]{article}
\usepackage[margin=1in]{geometry}

\usepackage{fontspec}
\setmainfont[BoldFont={HoeflerText-Bold},ItalicFont={Hoefler Text Italic},BoldItalicFont={Hoefler Text Black Italic},SmallCapsFont={HoeflerText-RomanSC},Ligatures={NoCommon, TeX}]{Hoefler Text}
\setmonofont[Ligatures={TeX}]{Source Code Pro Medium}

\usepackage[normalem]{ulem}

\title{Kids at the Skatepark}
\author{Colby Goettel}

\begin{document}
\maketitle

% 1) Audience Analysis
% Write one or two brief paragraphs analyzing your target audience and their values, beliefs, and assumptions. You are writing your paper for a general American audience: imagine that your argument will be distributed in a national newspaper or magazine. Within this general audience, you will need to identify who constitutes your specific audience. Who might find your claim initially unacceptable? What kinds of groups or individuals would care about your issue (e.g., educators, health care providers, government officials, parents, etc.)? What values, beliefs, or assumptions do they hold about your issue? Do you personally know any individuals who would disagree with your stance? What would their probable response to your argument be?

% target audience: parents
% audience values: safety

% Parents care for their children. That's what parents do. Yet parents bring their small children to skateparks and let them play on the ramps like slides. At the same time, skateboarders are rapidly moving throughout the park, trying difficult tricks, bailing, boards going everywhere. But there are children around.

The target audience, parents, have children. Parents want their children to be safe. They don't want bad things to happen to them. They assume that others will treat their children with respect and deference. They also bring their children to skateparks and let them play like it's a normal park. Sure, it's called a \textit{skatepark}, but it's definitely not a \textit{park}. It is an inherently dangerous place with people much larger than children moving rapidly. It is a danger that we take upon ourselves when we go to skate. We wear pads and helmets. We know how to bail so that we don't get too hurt. But children aren't familiar with these concepts. They are too young to understand. In fact, pre-teens at the skatepark are too young to understand social cues and frequently skate in the wrong areas~--- dangerous areas. Everyone is skating a certain line and children are playing around on that line. A gentle nudging can help, but anywhere outside of Utah and this can end in violence.

People who might find this claim initially unacceptable are, obviously, the parents who continue to bring their children to the skatepark. It's not like it's skaters bringing their kids, it's random moms bringing their children. People who have no idea about skate culture and etiquette. The people who would care about this issue are skaters, parents, and the general community. No one in any of these groups wants to hurt children. Skaters want to skate. Running into a child not only hurts the skater, but skaters trying to not hit a child runs the risk of getting seriously injured. I do not personally know anyone who disagrees with this stance because I'm part of the skate community, but I've had plenty of experience having to coach children and parents on skatepark etiquette. It is generally well-received because I try to present it in a calm way, but some skaters have been physically attacked by angry parents.

% YouTube video of skater getting punched by negligent mother

% 2) Consequential Issue Question
% Write out the WATCO question (What are the consequences of A on B?) that your paper will answer.
What are the consequences of having children at skateparks on their safety?

% 3) Pro/Con Analysis
% Summarize the arguments for and against your issue. You can write this pro/con analysis in the form of a bulleted list or as a paragraph.
Pros: safer children, safer skaters, skaters not getting hit by angry moms, more freedom to skate (in a skatepark), community better understands skateboarders.

Cons: community parks busier, parents might be frustrated at skaters, skateboarders ``hogging'' the skatepark, skaters not being sensitive to giving children a turn.

% 4) Preliminary Enthymeme
% Draft a preliminary enthymeme for your paper. This should include your claim, reason, and your assumption. Please follow the format below for writing your enthymeme (be sure to mark the terms and underline the verbs as shown):
Having children at skateparks (A) \uline{increases} (v1) the risk of hurting children (B) because having children at skateparks (A) \uline{increases} (v2) the likelihood of a child being hit by a skateboarder.

% Banning the advertising of junk food to children (A) decreases (v1) childhood obesity rates (B)
% b/c
% Banning the advertising of junk food to children (A) decreases (v2) calorie consumption from junk food among children (C).

\end{document}
