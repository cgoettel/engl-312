\documentclass[12pt]{article}
\usepackage[margin=1in]{geometry}
\usepackage{mathtools}
\usepackage{ifpdf}
\usepackage{mla}

\usepackage{fontspec}
% \setmainfont[BoldFont={HoeflerText-Bold},ItalicFont={Hoefler Text Italic},BoldItalicFont={Hoefler Text Black Italic},SmallCapsFont={HoeflerText-RomanSC},Ligatures={NoCommon, TeX}]{Hoefler Text}
\setmonofont[Ligatures={TeX},Scale=MatchLowercase]{Source Code Pro Medium}

\usepackage[normalem]{ulem}

\begin{document}
\begin{mla}{Colby}{Goettel}{Reed}{English 312}{2014-02-04}{Ruff drought}
% Think about open source code, crowd-sourcing, building on existing systems, allowing others to use your systems for their own growth. commits
GitHub is an on-line version control system for storing and maintaining code, and allowing others to build off your existing systems. It is not some small business idea that few people use, the big names are all present and accounted for including Google, Facebook, Yahoo, Microsoft, and all of their various projects. Because of this, it is not uncommon for developers, when applying for jobs, to put their GitHub account on their resume. Companies want to see what people do in their spare time, where their true passions lie, and GitHub is an excellent way for these companies to see how individuals spend their free time.

Because large and small companies, movers and shakers in IT, release their code open source, other developers are able to fork their projects and build off these existing systems. A common phrase in IT is that programming is 90\% finding someone else's code that does close to what you need and 10\% adapting it for your specific needs. Having code in public repositories allows others to gain insight into how systems are built and see a history of each project's commits (version control). It allows developers to fork projects and easily take off that first, difficult 90\% of programming. This greatly increases the speed that products can be prototyped, tested, revised, delivered, and maintained. Encouraging developers to release more of their code open source hurries the production of robust code. It improves software standards, creates a more secure world, and decreases the number of people needed to properly build and maintain systems.

% WATCO of providing open source code on creating more robust code? How does it affect it?

% building off existing systems.
On-line repositories like GitHub allow projects to be built by many developers at once (think Google Docs for programming). And when development of a certain project stops and others want to continue development, projects can be forked and built separately. GitHub also has a great feature called pull requests. In their own words: ``Pull requests let you tell others about changes you've pushed to a GitHub repository. Once a pull request is sent, interested parties can review the set of changes, discuss potential modifications, and even push follow-up commits if necessary.'' Code review is an essential part of building any system. A code review is done before a new version of a product is released and it involved checking for security holes, testing, and correcting any problems, all in conjunction with at least one other developer. Pull requests are a modern interpretation of code reviews that allow remote developers to co-develop.

% define robust code, talk about why it's essential. use cases of systems that did not have robust code.
% TRANSITION
When code is said to be robust, it is because it can handle major and minor functions of the systems to fail without bringing down the whole system. Robust code is security-minded and stable, it is not alpha or beta code that is not recommended for public consumption~--- it is the best of the best. It is what the general public expects their code to be. Having gone through rigorous testing, code becomes robust because bugs are fixed and users are able to test and give feedback.

% combine the thoughts of how building off existing systems creates robust code. talk about how obamacare and instagram show the ends of the spectrum.
But building robust code is increasingly difficult because of how quickly the IT landscape is changing. When developers can use others' code and tests, it allows faster and better products to be delivered.

In October 2010, Instagram, a photo and video sharing service, launched. A year and a half later, Facebook offered \$1B in cash and stock options to purchase Instagram. At the point it was acquired, Instagram had thirteen employees and thirty million customers. Contrast this example with the Affordable Care Act (ACA) web site that the federal government sunk billions of dollars into development, fifty-five government contractors (not fifty-five employees), and two years to build. And it didn't work on launch. These models show the ends of the spectrum and what is actually required to build scalable products in today's IT world.

% open source code


% people won't steal your ideas or your code. open source licenses. expound on MIT, Apache, and GPLv2y3 in particular. public and private repos


% connect open source to building off existing systems. github and bitbucket


% open source to robust. conclude


The entire business cycle is sped up because developers are able to work together to achieve a common goal. It is truly standing on the shoulders of giants.

\end{mla}
\end{document}

% A: Providing open source code
% v1: creates
% B: more robust code
% v2: enables
% C: developers to build off existing systems.

% Outline:
% Intro, contract
% C
% B
% C->B
% A
% procatalepsis
% A->C
% A->B (conclusion)
