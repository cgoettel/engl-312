\documentclass[12pt]{article}
\usepackage[margin=1in]{geometry}

\usepackage{fontspec}
\setmainfont[BoldFont={HoeflerText-Bold},ItalicFont={Hoefler Text Italic},BoldItalicFont={Hoefler Text Black Italic},SmallCapsFont={HoeflerText-RomanSC},Ligatures={NoCommon, TeX}]{Hoefler Text}
\setmonofont[Ligatures={TeX}]{Source Code Pro Medium}

\usepackage[normalem]{ulem}

\title{The US as a Terrorist Organization}
\author{Colby Goettel}

\begin{document}
\maketitle

% 1) Audience Analysis
% Write one or two brief paragraphs analyzing your target audience and their values, beliefs, and assumptions. You are writing your paper for a general American audience: imagine that your argument will be distributed in a national newspaper or magazine. Within this general audience, you will need to identify who constitutes your specific audience. Who might find your claim initially unacceptable? What kinds of groups or individuals would care about your issue (e.g., educators, health care providers, government officials, parents, etc.)? What values, beliefs, or assumptions do they hold about your issue? Do you personally know any individuals who would disagree with your stance? What would their probable response to your argument be?

% target audience: us citizens, governments, students, general populace.
% audience values: freedom, privacy, protection from those who wish to do them harm.

The audience includes US citizens, the US government, foreign governments, students, and the general populace. US citizens and the general populace care about freedom, privacy, and protection from those who wish to do them harm. Students are generally more interested in politics than the rest of the population and therefore are an integral part of this discussion. Students are generally the ones to revolt. Governments should, theoretically, have the best interests of their citizenry in mind.

This claim is initially unacceptable to many, many people within and without the United States. This is not a right- or left-wing issue. The stakeholders include US citizens, foreign citizens, and both allies and enemies. The current scope of US intrusion is global. These stakeholders have the same beliefs and values as the target audience, viz.: freedom, privacy, and protection. I know many who disagree with this view, either to a degree or entirely. Their response is generally to take off my tinfoil hat.

% 2) Consequential Issue Question
% Write out the WATCO question (What are the consequences of A on B?) that your paper will answer.
What are the consequences of the US government committing terrorist actions, foreign and domestic, on US citizens.

% 3) Pro/Con Analysis
% Summarize the arguments for and against your issue. You can write this pro/con analysis in the form of a bulleted list or as a paragraph.
Pros: public awareness, public and government action, increased privacy and protection.

Cons: getting put on a federal watch list.

% 4) Preliminary Enthymeme
% Draft a preliminary enthymeme for your paper. This should include your claim, reason, and your assumption. Please follow the format below for writing your enthymeme (be sure to mark the terms and underline the verbs as shown):
The US government (A) \uline{is} (v1) a terrorist organization (B) because the US government (A) \uline{fits} (v2) their own description of a terrorist organization.

% Definitions:
%     http://www.fbi.gov/stats-services/publications/terrorism-2002-2005
%     http://www.uscis.gov/iframe/ilink/docView/SLB/HTML/SLB/0-0-0-1/0-0-0-29/0-0-0-5017.html

% Banning the advertising of junk food to children (A) decreases (v1) childhood obesity rates (B)
% b/c
% Banning the advertising of junk food to children (A) decreases (v2) calorie consumption from junk food among children (C).

\end{document}
